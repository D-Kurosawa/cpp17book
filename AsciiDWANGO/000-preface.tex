%
% prologue
\hyperchaptern{prologue}{はじめに}

本書は2017年に規格制定されたプログラミング言語C++の国際規格、ISO/IEC
14882:2017の新機能をほぼすべて解説している。

新しいC++17は不具合を修正し、プログラマーの日々のコーディングを楽にする新機能がいくつも追加された。その結果、C++の特徴であるパフォーマンスや静的型付けは損なうことなく、近年の動的な型の弱い言語に匹敵するほどの柔軟な記述を可能にしている。

人によっては、新機能を学ぶのは労多くして益少なしと考えるかもしれぬが、C++の新機能は現実の問題を解決するための便利な道具として追加されるもので、仮に機能を使わないとしても問題はなくならないため、便利な道具なく問題に対処しなければならぬ。また、C++の機能は一般的なプログラマーにとって自然だと感じるように設計されているため、利用は難しくない。もしC++が難しいと感じるのであれば、それはC++が解決すべき現実の問題が難しいのだ。なんとなれば、我々は理想とは程遠い歪なアーキテクチャのコンピューターを扱う時代に生きている。CPUの性能上昇は停滞し、メモリはCPUに比べて遥かに遅く、しかもそのアクセスは定数時間ではない。キャッシュに収まる局所性を持つデータへの操作は無料同然で、キャッシュサイズの単位はすでにMBで数えられている。手のひらに乗る超低電力CPUでさえマルチコアが一般的になり、並列処理、非同期処理は全プログラマーが考慮せねばならぬ問題になった。

そのような時代にあたっては、かつては最良であった手法はその価値を失い、あるいは逆に悪い手法と成り下がる。同時に昔は現実的ではなかった手法が今ではかえってまともな方法になることさえある。このため、現在活発に使われている生きている言語は、常に時代に合わない機能を廃止し、必要な機能を追加する必要がある。C++の発展はここで留まることなく、今後もC++が使われ続ける限り、修正と機能追加が行われていくだろう。

本書の執筆はGithub上で公開して行われた。
\vskip 0.5zw
\hspace*{1em}\url{https://github.com/EzoeRyou/cpp17book}
\vskip 0.5zw
本書のライセンスはGPLv3だ。

本書の執筆では株式会社ドワンゴとGitHub上でPull
Requestを送ってくれた多くの貢献者の協力によって、誤りを正し、より良い記述を実現できた。この場を借りて謝意を表したい。

本書に誤りを見つけたならば、Pull
Requestを送る先は
\vskip 0.5zw
\hspace*{1em}\url{https://github.com/EzoeRyou/cpp17book}
\vskip 0.5zw
\noindent
だ。

\vskip 1zw
\begin{flushright}
江添亮
\end{flushright}


%
% Preface
\cleardoublepage
\hyperchaptern{chapter0}{序}

%
% Section 0.1
\hypersection{section0-1}{C++の規格}

プログラミング言語C++はISOの傘下で国際規格ISO/IEC
14882として制定されている。この規格は数年おきに改定されている。一般にC++の規格を参照するときは、規格が制定した西暦の下二桁を取って、C++98(1998年発行)とかC++11(2011年発行)と呼ばれている。現在発行されているC++の規格は以下のとおり。
\index{ぷろぐらみんぐげんごC++@プログラミング言語C++}
\index{ISO/IEC 14882}

%
% SubSection 0.1.1
\hypersubsection{section0-1-1}{C++98}
\index{C++98}

C++98は1998年に制定された最初のC++の規格である。本来ならば1994年か1995年には制定させる予定が大幅にずれて、1998年となった。

%
% SubSection 0.1.2
\hypersubsection{section0-1-2}{C++03}
\index{C++03}

C++03はC++98の文面の曖昧な点を修正したマイナーアップデートとして2003年に制定された。新機能の追加はほとんどない。

%
% SubSection 0.1.3
\hypersubsection{section0-1-3}{C++11}
\index{C++11}

C++11は制定途中のドラフト段階では元C++0xと呼ばれていた。これは、200x年までに規格が制定される予定だったからだ。予定は大幅に遅れ、ようやく規格が制定されたときにはすでに2011年の年末になっていた。C++11ではとても多くの新機能が追加された。

%
% SubSection 0.1.4
\hypersubsection{section0-1-4}{C++14}
\index{C++14}

C++14は2014年に制定された。C++11の文面の誤りを修正した他、少し新機能が追加された。本書で解説する。

%
% SubSection 0.1.4
\hypersubsection{section0-1-4}{C++17}
\index{C++17}

C++17は2017年に制定されることが予定されている最新のC++規格で、本書で解説する。

%
% Section 0.2
\hypersection{section0-2}{C++の将来の規格}

%
% SubSection 0.2.1
\hypersubsection{section0-2-1}{C++20}
\index{C++20}

C++20は2020年に制定されることが予定されている次のC++規格だ。この規格では、モジュール、コンセプト、レンジ、ネットワークに注力することが予定されている。

%
% Section 0.3
\hypersection{section0-3}{コア言語とライブラリ}

C++の標準規格は、大きく分けて、Cプリプロセッサーとコア言語とライブラリからなる。

Cプリプロセッサーとは、C++がC言語から受け継いだ機能だ。ソースファイルをトークン列単位で分割して、トークン列の置換ができる。
\index{Cぷりぷろせつさ@Cプリプロセッサー}

コア言語とは、ソースファイルに書かれたトークン列の文法とその意味のことだ。
\index{こあげんご@コア言語}

ライブラリとは、コア言語機能を使って実装されたもので、標準に提供されているものだ。標準ライブラリには、純粋にコア言語の機能のみで実装できるものと、それ以外の実装依存の方法やコンパイラーマジックが必要なものとがある。
\index{らいぶらり@ライブラリ}
