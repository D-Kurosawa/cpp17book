%
% Section 3.22
\hypersection{section3-22}{可変長using宣言}
\index{かへんちようusingせんげん@可変長\texttt{using}宣言}

この機能は超上級者向けだ。

C++17では\lstinline!using!宣言をコンマで区切ることができるようになった。

\begin{lstlisting}[language=C++]
int x, y ;

int main()
{
    using ::x, ::y ;
}
\end{lstlisting}

これは、C++14で
\begin{lstlisting}[language=C++]
using ::x ;
using ::y ;
\end{lstlisting}
と書くのと等しい。

C++17では、\lstinline!using!宣言でパック展開ができるようになった。この機能に正式な名前は付いていないが、可変長\lstinline!using!宣言(Variadic
using declaration)と呼ぶのがわかりやすい。

\begin{lstlisting}[language=C++]
template < typename ... Types >
struct S : Types ...
{
    using Types::operator() ... ;
    void operator () ( long ) { }
} ;


struct A
{
    void operator () ( int ) { }
} ;

struct B
{
    void operator () ( double ) { }
} ;

int main()
{
    S<A, B> s ;
    s(0) ; // A::operator()
    s(0L) ; // S::operator()
    s(0.0) ; // B::operator()
}
\end{lstlisting}

機能テストマクロは~\lstinline!__cpp_variadic_using!, 値は201611。
\index{\_\_cpp\_variadic\_using@\texttt{\_\_cpp\_variadic\_using}}
