%
% Section 9.26
\hypersection{section9-26}{最大公約数(gcd)と最小公倍数(lcm)}

C++17ではヘッダーファイル~\lstinline!<numeric>!~に最大公約数(\lstinline!gcd!)と最小公倍数(\lstinline!lcm!)が追加された。
\index{<numeric>@\texttt{<numeric>}}

\begin{lstlisting}[language=C++]
int main()
{
    int a, b ;

    while( std::cin >> a >> b )
    {
        std::cout
            << "gcd: " << gcd(a,b)
            << "\nlcm: " << lcm(a,b) << '\n' ;
    }
}
\end{lstlisting}

%
% SubSection 9.26.1
\hypersubsection{section9-26-1}{gcd : 最大公約数}
\index{gcd@\texttt{gcd}}\index{さいだいこうやくすう@最大公約数}

\bgroup
\begin{lstlisting}[language=C++]
template <class M, class N>
constexpr std::common_type_t<M,N> gcd(M m, N n)
{
    if ( n == 0 )
        return m ;
    else
        return gcd( n, std::abs(m) % std::abs(n) ) ; 
}
\end{lstlisting}
\egroup

\lstinline!gcd(m, n)!は\lstinline!m!と\lstinline!n!がともにゼロの場合ゼロを返す。それ以外の場合、\(|m|\)と\(|n|\)の最大公約数(Greatest
Common Divisor)を返す。

%
% SubSection 9.26.2
\hypersubsection{section9-26-2}{lcm : 最小公倍数}
\index{lcm@\texttt{lcm}}\index{さいしようこうばいすう@最小公倍数}

\bgroup
\begin{lstlisting}[language=C++]
template <class M, class N>
constexpr std::common_type_t<M,N> lcm(M m, N n)
{
    if ( m == 0 || n == 0 )
        return 0 ;
    else
        return std::abs(m) / gcd( m, n ) * std::abs(n) ;
}
\end{lstlisting}
\egroup

\lstinline!lcm(m,n)!は、\lstinline!m!と\lstinline!n!のどちらかがゼロの場合ゼロを返す。それ以外の場合、\(|m|\)と\(|n|\)の最小公倍数(Least
Common Multiple)を返す。
