%
% Section 9.11
\hypersection{section9-11}{メモリー管理アルゴリズム}

C++17ではヘッダーファイル~\lstinline!<memory>!~にメモリー管理用のアルゴリズムが追加された。
\index{<memory>@\texttt{<memory>}}\index{めもりかんりあるごりずむ@メモリー管理アルゴリズム}

%
% SubSection 9.11.1
\hypersubsection{section9-11-1}{addressof}
\index{addressof@\texttt{addressof}}

\bgroup
\begin{lstlisting}[language=C++]
template <class T> constexpr T* addressof(T& r) noexcept;
\end{lstlisting}
\egroup

\lstinline!addressof!はC++17以前からもある。\lstinline!addressof(r)!は\lstinline!r!のポインターを取得する。たとえ、\lstinline!r!の型が\lstinline!operator &!をオーバーロードしていても正しいポインターを取得できる。

\begin{lstlisting}[language=C++]
struct S
{
    S * operator &() const noexcept
    { return nullptr ; } 
} ;

int main()
{
    S s ;

    // nullptr
    S * p1 = & s ;
    // 妥当なポインター
    S * p2 = std::addressof(s) ;

}
\end{lstlisting}

%
% SubSection 9.11.2
\hypersubsection{section9-11-2}{uninitialized\texttt{\_}default\texttt{\_}construct}
\index{uninitialized\_default\_construct@\texttt{uninitialized\_default\_construct}}

\bgroup
\begin{lstlisting}[language=C++]
template <class ForwardIterator>
void uninitialized_default_construct(
    ForwardIterator first, ForwardIterator last);

template <class ForwardIterator, class Size>
ForwardIterator uninitialized_default_construct_n(
    ForwardIterator first, Size n);
\end{lstlisting}
\egroup

\lstinline![first, last)!の範囲、もしくは\lstinline!first!から\lstinline!n!個の範囲の未初期化のメモリーを\lstinline!typename iterator_traits<ForwardIterator>::value_type!でデフォルト初期化する。2つ目のアルゴリズムは\lstinline!first!から\lstinline!n!個をデフォルト初期化する。

\begin{lstlisting}[language=C++]
int main()
{
    std::shared_ptr<void> raw_ptr
    (   ::operator new( sizeof(std::string) * 10 ),
        [](void * ptr){ ::operator delete(ptr) ; } ) ;
 
    std::string * ptr = static_cast<std::string *>( raw_ptr.get() ) ;

    std::uninitialized_default_construct_n( ptr, 10 ) ;
    std::destroy_n( ptr, 10 ) ;
}
\end{lstlisting}

%
% SubSection 9.11.3
\hypersubsection{section9-11-3}{uninitialized\texttt{\_}value\texttt{\_}construct}
\index{uninitialized\_value\_construct@\texttt{uninitialized\_value\_construct}}

\bgroup
\begin{lstlisting}[language=C++]
template <class ForwardIterator>
void uninitialized_value_construct(
    ForwardIterator first, ForwardIterator last);

template <class ForwardIterator, class Size>
ForwardIterator uninitialized_value_construct_n(
    ForwardIterator first, Size n);
\end{lstlisting}
\egroup

使い方は\lstinline!uninitialized_default_construct!と同じ。ただし、こちらはデフォルト初期化ではなく値初期化する。

%
% SubSection 9.11.4
\hypersubsection{section9-11-4}{uninitialized\texttt{\_}copy}
\index{uninitialized\_copy@\texttt{uninitialized\_copy}}

\bgroup
\begin{lstlisting}[language=C++]
template <class InputIterator, class ForwardIterator>
ForwardIterator
uninitialized_copy( InputIterator first, InputIterator last,
                    ForwardIterator result);

template <class InputIterator, class Size, class ForwardIterator>
ForwardIterator
uninitialized_copy_n(   InputIterator first, Size n,
                        ForwardIterator result);
\end{lstlisting}
\egroup

\lstinline![first, last)!の範囲、もしくは\lstinline!first!から\lstinline!n!個の範囲の値を、\lstinline!result!の指す未初期化のメモリーにコピー構築する。

\begin{lstlisting}[language=C++]
int main()
{
    std::vector<std::string> input(10, "hello") ;

    std::shared_ptr<void> raw_ptr
    (   ::operator new( sizeof(std::string) * 10 ),
        [](void * ptr){ ::operator delete(ptr) ; } ) ;
 
    std::string * ptr = static_cast<std::string *>( raw_ptr.get() ) ;


    std::uninitialized_copy_n( std::begin(input), 10, ptr ) ;
    std::destroy_n( ptr, 10 ) ;
}
\end{lstlisting}

%
% SubSection 9.11.5
\hypersubsection{section9-11-5}{uninitialized\texttt{\_}move}
\index{uninitialized\_move@\texttt{uninitialized\_move}}

\bgroup
\begin{lstlisting}[language=C++]
template <class InputIterator, class ForwardIterator>
ForwardIterator
uninitialized_move( InputIterator first, InputIterator last,
                    ForwardIterator result);

template <class InputIterator, class Size, class ForwardIterator>
pair<InputIterator, ForwardIterator>
uninitialized_move_n(   InputIterator first, Size n,
                        ForwardIterator result);
\end{lstlisting}
\egroup

使い方は\lstinline!uninitialized_copy!と同じ。ただしこちらはコピーではなくムーブする。

%
% SubSection 9.11.6
\hypersubsection{section9-11-6}{uninitialized\texttt{\_}fill}
\index{uninitialized\_fill@\texttt{uninitialized\_fill}}

\bgroup
\begin{lstlisting}[language=C++]
template <class ForwardIterator, class T>
void uninitialized_fill(
    ForwardIterator first, ForwardIterator last,
    const T& x);

template <class ForwardIterator, class Size, class T>
ForwardIterator uninitialized_fill_n(
    ForwardIterator first, Size n,
    const T& x);
\end{lstlisting}
\egroup

\lstinline![first, last)!の範囲、もしくは\lstinline!first!から\lstinline!n!個の範囲の未初期化のメモリーを、コンストラクターに実引数\lstinline!x!を与えて構築する。

%
% SubSection 9.11.7
\hypersubsection{section9-11-7}{destroy}
\index{destroy@\texttt{destroy}}

\bgroup
\begin{lstlisting}[language=C++]
template <class T>
void destroy_at(T* location);
\end{lstlisting}
\egroup

\lstinline!location->~T()!を呼び出す。

\begin{lstlisting}[language=C++]
template <class ForwardIterator>
void destroy(ForwardIterator first, ForwardIterator last);

template <class ForwardIterator, class Size>
ForwardIterator destroy_n(ForwardIterator first, Size n);
\end{lstlisting}

\lstinline![first, last)!の範囲、もしくは\lstinline!first!から\lstinline!n!個の範囲に\lstinline!destroy_at!を呼び出す。
