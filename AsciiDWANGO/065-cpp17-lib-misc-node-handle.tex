%
% Section 9.19
\hypersection{section9-19}{連想コンテナーへのsplice操作}

C++17では、連想コンテナーと非順序連想コンテナーで\lstinline!splice!操作がサポートされた。
\index{れんそうこんてな@連想コンテナー}\index{ひじゆんじよれんそうこんてな@非順序連想コンテナー}\index{splice@\texttt{splice}}

対象のコンテナーは\lstinline!map!, \lstinline!set!,
\lstinline!multimap!, \lstinline!multiset!, \lstinline!unordered_map!,
\lstinline!unordered_set!, \lstinline!unordered_multimap!,
\lstinline!unordered_multiset!だ。
\index{map@\texttt{map}}\index{set@\texttt{set}}\index{multimap@\texttt{multimap}}\index{multiset@\texttt{multiset}}\index{unordered\_map@\texttt{unordered\_map}}\index{unordered\_set@\texttt{unordered\_set}}\index{unordered\_multimap@\texttt{unordered\_multimap}}\index{unordered\_multiset@\texttt{unordered\_multiset}}

\lstinline!splice!操作とは\lstinline!list!で提供されている操作で、アロケーター互換の\lstinline!list!のオブジェクトの要素をストレージと所有権ごと別のオブジェクトに移動する機能だ。

\begin{lstlisting}[language=C++]
int main()
{
    std::list<int> a = {1,2,3} ;
    std::list<int> b = {4,5,6} ;

    a.splice( std::end(a), b, std::begin(b) ) ;

    // aは{1,2,3,4}
    // bは{5,6}

    b.splice( std::end(b), a ) ;

    // aは{}
    // bは{5,6,1,2,3,4}

}
\end{lstlisting}

連想コンテナーでは、ノードハンドルという仕組みを用いて、コンテナーのオブジェクトから要素の所有権をコンテナーの外に出す仕組みで、\lstinline!splice!操作を行う。

%
% SubSection 9.19.1
\hypersubsection{section9-19-1}{merge}

すべての連想コンテナーと非順序連想コンテナーは、メンバー関数\lstinline!merge!を持っている。コンテナー\lstinline!a!,
\lstinline!b!がアロケーター互換のとき、\lstinline!a.merge(b)!は、コンテナー\lstinline!b!の要素の所有権をすべてコンテナー\lstinline!a!に移す。
\index{れんそうこんてな@連想コンテナー!merge@\texttt{merge}}\index{ひじゆんじよれんそうこんてな@非順序連想コンテナー!merge@\texttt{merge}}\index{merge@\texttt{merge}}

\begin{lstlisting}[language=C++]
int main()
{
    std::set<int> a = {1,2,3} ;
    std::set<int> b = {4,5,6} ;

    // bの要素をすべてaに移す
    a.merge(b) ;

    // aは{1,2,3,4,5,6}
    // bは{}
}
\end{lstlisting}

もし、キーの重複を許さないコンテナーの場合で、値が重複した場合、重複した要素は移動しない。

\begin{lstlisting}[language=C++]
int main()
{
    std::set<int> a = {1,2,3} ;
    std::set<int> b = {1,2,3,4,5,6} ;

    a.merge(b) ;

    // aは{1,2,3,4,5,6}
    // bは{1,2,3}

}
\end{lstlisting}

\lstinline!merge!によって移動された要素を指すポインターとイテレーターは、要素の移動後も妥当である。ただし、所属するコンテナーのオブジェクトが変わる。

\begin{lstlisting}[language=C++]
int main()
{
    std::set<int> a = {1,2,3} ;
    std::set<int> b = {4,5,6} ;


    auto iterator = std::begin(b) ;
    auto pointer = &*iterator ;

    a.merge(b) ;

    // iteratorとpointerはまだ妥当
    // ただし要素はaに所属する
}
\end{lstlisting}

%
% SubSection 9.19.2
\hypersubsection{section9-19-2}{ノードハンドル}

ノードハンドルとは、コンテナーオブジェクトから要素を構築したストレージの所有権を切り離す機能だ。
\index{のどはんどる@ノードハンドル}

ノードハンドルの型は、各コンテナーのネストされた型名\lstinline!node_type!となる。たとえば~\lstinline!std::set<int>!~のノードハンドル型は、\lstinline!std::set<int>::node_type!となる。
\index{node\_type@\texttt{node\_type}}\index{のどはんどる@ノードハンドル!node\_type@\texttt{node\_type}}

ノードハンドルは以下のようなメンバーを持っている。

\begin{lstlisting}[language=C++]
class node_handle
{
public :
    // ネストされた型名
    using value_type = ... ;        // set限定、要素型
    using key_type = ... ;          // map限定、キー型
    using mapped_type = ... ;       // map限定、マップ型
    using allocator_type = ... ;    // アロケーターの型

    // デフォルトコンストラクター
    // ムーブコンストラクター
    // ムーブ代入演算子


    // 値へのアクセス
    value_type & value() const ;   // set限定
    key_type & key() const ;        // map限定
    mapped_type & mapped() const ;  // map限定

    // アロケーターへのアクセス
    allocator_type get_allocator() const ;

    // 空かどうかの判定
    explicit operator bool() const noexcept ;
    bool empty() const noexcept ;

    void swap( node_handle & ) ;
} ;
\end{lstlisting}

\lstinline!set!のノードハンドルはメンバー関数\lstinline!value!で値を得る。
\index{value@\texttt{value}}\index{set@\texttt{set}!value@\texttt{value}}\index{のどはんどる@ノードハンドル!value@\texttt{value}}

\begin{lstlisting}[language=C++]
int main()
{
    std::set<int> c = {1,2,3} ;

    auto n = c.extract(2) ;

    // n.value() == 2
    // cは{1,3}
}
\end{lstlisting}

\lstinline!map!のノードハンドルはメンバー関数\lstinline!key!と\lstinline!mapped!でそれぞれの値を得る。
\index{key@\texttt{key}}\index{map@\texttt{map}!key@\texttt{key}}\index{のどはんどる@ノードハンドル!key@\texttt{key}}\index{mapped@\texttt{mapped}}\index{map@\texttt{map}!mapped@\texttt{mapped}}\index{のどはんどる@ノードハンドル!mapped@\texttt{mapped}}

\begin{lstlisting}[language=C++]
int main()
{
    std::map< int, int > m =
    {
        {1,1}, {2,2}, {3,3}
    } ;

    auto n = m.extract(2) ;

    // n.key() == 2 
    // n.mapped() == 2
    // mは{{1,1},{3,3}}

}
\end{lstlisting}

ノードハンドルはノードをコンテナーから切り離し、所有権を得る。そのため、ノードハンドルによって得たノードは、元のコンテナーから独立し、元のコンテナーオブジェクトの破棄の際にも破棄されない。ノードハンドルのオブジェクトの破棄時に破棄される。このため、ノードハンドルはアロケーターのコピーも持つ。

\begin{lstlisting}[language=C++]
int main()
{
    std::set<int>::node_type n ;

    {
        std::set<int> c = { 1,2,3 } ;
        // 所有権の移動
        n = c.extract( std::begin(c) ) ;
        // cが破棄される
    }

    // OK
    // ノードハンドルによって所有権が移動している
    int x = n.value() ;

    // nが破棄される
}
\end{lstlisting}

%
% SubSection 9.19.3
\hypersubsection{section9-19-3}{extract : ノードハンドルの取得}
\index{extract@\texttt{extract}}\index{のどはんどる@ノードハンドル!extract@\texttt{extract}}\index{のどはんどる@ノードハンドル!しゆとく@取得}

\bgroup
\begin{lstlisting}[language=C++]
node_type extract( const_iterator position ) ;
node_type extract( const key_type & x ) ;
\end{lstlisting}
\egroup

連想コンテナーと非順序連想コンテナーのメンバー関数\lstinline!extract!は、ノードハンドルを取得するためのメンバー関数だ。

メンバー関数\lstinline!extract(position)!は、イテレーターの\lstinline!position!が指す要素を、コンテナーから除去して、その要素を所有するノードハンドルを返す。

\begin{lstlisting}[language=C++]
int main()
{
    std::set<int> c = {1,2,3} ;

    auto n1 = c.extract( std::begin(c) ) ;

    // cは{2,3}

    auto n2 = c.extract( std::begin(c) ) ;

    // cは{3}

}
\end{lstlisting}

メンバー関数\lstinline!extract(x)!は、キー\lstinline!x!がコンテナーに存在する場合、その要素をコンテナーから除去して、その要素を所有するノードハンドルを返す。存在しない場合、空のノードハンドルを返す。

\begin{lstlisting}[language=C++]
int main()
{
    std::set<int> c = {1,2,3} ;

    auto n1 = c.extract( 1 ) ;
    // cは{2,3}

    auto n2 = c.extract( 2 ) ;
    // cは{3}

    // キー4は存在しない
    auto n3 = c.extract( 4 ) ;
    // cは{3}
    // n3.empty() == true
}
\end{lstlisting}

キーの重複を許すコンテナーの場合、複数あるうちの1つの所有権が解放される。

\begin{lstlisting}[language=C++]
int main()
{
    std::multiset<int> c = {1,1,1} ;
    auto n = c.extract(1) ;
    // cは{1,1}
}
\end{lstlisting}

%
% SubSection 9.19.4
\hypersubsection{section9-19-4}{insert : ノードハンドルから要素の追加}
\index{insert@\texttt{insert}}\index{のどはんどる@ノードハンドル!insert@\texttt{insert}}\index{のどはんどる@ノードハンドル!ようそのついか@要素の追加}

\bgroup
\begin{lstlisting}[language=C++]
// キーの重複を許さないコンテナーの場合
insert_return_type  insert(node_type&& nh);
// キーの重複を許すmultiコンテナーの場合
iterator  insert(node_type&& nh);

// ヒント付きのinsert
iterator            insert(const_iterator hint, node_type&& nh);
\end{lstlisting}
\egroup

ノードハンドルをコンテナーのメンバー関数\lstinline!insert!の実引数に渡すと、ノードハンドルから所有権をコンテナーに移動する。

\begin{lstlisting}[language=C++]
int main()
{
    std::set<int> a = {1,2,3} ;
    std::set<int> b = {4,5,6} ;

    auto n = a.extract(1) ;

    b.insert( std::move(n) ) ;

    // n.empty() == true
}
\end{lstlisting}

ノードハンドルが空の場合、何も起こらない。

\begin{lstlisting}[language=C++]
int main()
{
    std::set<int> c ;
    std::set<int>::node_type n ;

    // 何も起こらない
    c.insert( std::move(n) ) ;
}
\end{lstlisting}

キーの重複を許さないコンテナーに、すでにコンテナーに存在するキーと等しい値を所有するノードハンドルを\lstinline!insert!しようとすると、\lstinline!insert!は失敗する。

\begin{lstlisting}[language=C++]
int main()
{
    std::set<int> c = {1,2,3} ;

    auto n = c.extract(1) ;
    c.insert( 1 ) ;

    // 失敗する
    c.insert( std::move(n) ) ; 
}
\end{lstlisting}

第一引数にイテレーター\lstinline!hint!を受け取る\lstinline!insert!の挙動は、従来の\lstinline!insert!と同じだ。要素が\lstinline!hint!の直前に追加されるのであれば償却定数時間で処理が終わる。

ノードハンドルを実引数に受け取る\lstinline!insert!の戻り値の型は、キーの重複を許す\lstinline!multi!コンテナーの場合\lstinline!iterator!。キーの重複を許さないコンテナーの場合、\lstinline!insert_return_type!となる。

\lstinline!multi!コンテナーの場合、戻り値は追加した要素を指すイテレーターとなる。

\begin{lstlisting}[language=C++]
int main()
{
    std::multiset<int> c { 1,2,3 } ;

    auto n = c.extract( 1 ) ;

    auto iter = c.insert( n ) ;

    // cは{1,2,3}
    // iterは1を指す
}
\end{lstlisting}

キーの重複を許さないコンテナーの場合、コンテナーにネストされた型名\lstinline!insert_return_type!が戻り値の型となる。たとえば\lstinline!set<int>!の場合、\lstinline!set<int>::insert_return_type!となる。

\lstinline!insert_return_type!の具体的な名前は規格上規定されていない。\lstinline[breaklines=true]!insert_return_ type!は以下のようなデータメンバーを持つ型となっている。
\index{insert\_return\_type@\texttt{insert\_return\_type}}\index{のどはんどる@ノードハンドル!insert\_return\_type@\texttt{insert\_return\_type}}

\begin{lstlisting}[language=C++]
struct insert_return_type
{
    iterator position ;
    bool inserted ;
    node_type node ;
} ;
\end{lstlisting}

\lstinline!position!は\lstinline!insert!によってコンテナーに所有権を移動して追加された要素を指すイテレーター、\lstinline!inserted!は要素の追加が行われた場合に\lstinline!true!となる\lstinline!bool!,
\lstinline!node!は要素の追加が失敗したときにノードハンドルの所有権が移動されるノードハンドルとなる。

\lstinline!insert!に渡したノードハンドルが空のとき、\lstinline!inserted!は\lstinline!false!,
\lstinline!position!は\lstinline!end()!, \lstinline!node!は空になる。

\begin{lstlisting}[language=C++]
int main()
{
    std::set<int> c = {1,2,3} ;
    std::set<int>::node_type n ; // 空

    auto [position, inserted, node] = c.insert( std::move(n) ) ;

    // inserted == false
    // position == c.end()
    // node.empty() == true
}
\end{lstlisting}

\lstinline!insert!が成功したとき、\lstinline!inserted!は\lstinline!true!,
\lstinline!position!は追加された要素を指し、\lstinline!node!は空になる。

\begin{lstlisting}[language=C++]
int main()
{
    std::set<int> c = {1,2,3} ;
    auto n = c.extract(1) ;

    auto [position, inserted, node] = c.insert( std::move(n) ) ;

    // inserted == true
    // position == c.find(1)
    // node.empty() == true
}
\end{lstlisting}

\lstinline!insert!が失敗したとき、つまりすでに同一のキーがコンテナーに存在したとき、\lstinline!inserted!は\lstinline!false!,
\lstinline!node!は\lstinline!insert!を呼び出す前のノードハンドルの値、\lstinline!position!はコンテナーの中の追加しようとしたキーに等しい要素を指す。\lstinline!insert!に渡したノードハンドルは未規定の値になる。

\begin{lstlisting}[language=C++]
int main()
{
    std::set<int> c = {1,2,3} ;
    auto n = c.extract(1) ;
    c.insert(1) ;

    auto [position, inserted, node] = c.insert( std::move(n) ) ;

    // nは未規定の値
    // inserted == false
    // nodeはinsert( std::move(n) )を呼び出す前のnの値
    // position == c.find(1)
}
\end{lstlisting}

規格はこの場合の\lstinline!n!の値について規定していないが、最もありうる実装としては、\lstinline!n!は\lstinline!node!にムーブされるので、\lstinline!n!は空になり、ムーブ後の状態になる。

%
% SubSection 9.19.5
\hypersubsection{section9-19-5}{ノードハンドルの利用例}

ノードハンドルの典型的な使い方は以下のとおり。

%
% SubsubSection 9.19.5.1
\noindent
\hypersubsubsection{section9-19-5-1}{ストレージの再確保なしに、コンテナーの一部の要素だけ別のコンテナーに移す}

\bgroup
\begin{lstlisting}[language=C++]
int main()
{
    std::set<int> a = {1,2,3} ;
    std::set<int> b = {4,5,6} ;

    auto n = a.extract(1) ;
    b.insert( std::move(n) ) ;
}
\end{lstlisting}
\egroup

%
% SubsubSection 9.19.5.2
\noindent
\hypersubsubsection{section9-19-5-2}{コンテナーの寿命を超えて要素を存続させる}

\bgroup
\begin{lstlisting}[language=C++]
int main()
{
    std::set<int>::node_type n ;

    {
        std::set<int> c = {1,2,3} ;
        n = c.extract(1) ;
        // cが破棄される
    }

    // コンテナーの破棄後も存続する
    int value = n.value() ;
}
\end{lstlisting}
\egroup

%
% SubsubSection 9.19.5.3
\noindent
\hypersubsubsection{section9-19-5-3}{mapのキーを変更する}

\lstinline!map!ではキーは変更できない。キーを変更したければ、元の要素は削除して、新しい要素を追加する必要がある。これには動的なストレージの解放と確保が必要になる。

ノードハンドルを使えば、既存の要素のストレージに対して、所有権を\lstinline!map!から引き剥がした上で、キーを変更して、もう一度\lstinline!map!に差し戻すことができる。

\begin{lstlisting}[language=C++]
int main()
{
    std::map< std::string, std::string > m =
    {
        {"cat", "meow"},
        {"DOG", "bow"}, // キーを間違えたので変更したい
        {"cow", "moo"}
    } ;

    // 所有権を引き剥がす
    auto n = m.extract("DOG") ;
    // キーを変更
    n.key() = "dog" ;
    // 差し戻す
    m.insert( std::move(n) ) ;
}
\end{lstlisting}

