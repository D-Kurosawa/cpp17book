%
% Section 9.10
\hypersection{section9-10}{not\texttt{\_}fn : 戻り値の否定ラッパー}

\lstinline!not_fn!はヘッダーファイル~\lstinline!<functional>!~で定義されている。
\index{not\_fn@\texttt{not\_fn}}\index{<functional>@\texttt{<functional>}}

\begin{lstlisting}[language=C++]
template <class F> unspecified not_fn(F&& f);
\end{lstlisting}

関数オブジェクト\lstinline!f!に対して\lstinline!not_fn(f)!を呼び出すと、戻り値として何らかの関数オブジェクトが返ってくる。その関数オブジェクトを呼び出すと、実引数を\lstinline!f!に渡して\lstinline!f!を関数呼び出しして、戻り値を~\lstinline"operator !"~で否定して返す。

\begin{lstlisting}[language=C++]
int main()
{

    auto r1 = std::not_fn( []{ return true ; } ) ;

    r1() ; // false

    auto r2 = std::not_fn( []( bool b ) { return b ; } ) ;

    r2(true) ; // false
}
\end{lstlisting}

すでに廃止予定になった\lstinline!not1!, \lstinline!not2!の代替品。
\index{not1@\texttt{not1}}\index{not2@\texttt{not2}}
