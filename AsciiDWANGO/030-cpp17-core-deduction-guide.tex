%
% Section 3.16
\hypersection{section3-16}{クラステンプレートのコンストラクターからの実引数推定}

C++17ではクラステンプレートのコンストラクターの実引数からテンプレート実引数の推定が行えるようになった。
\index{てんぷれとじつひきすうのすいてい@テンプレート実引数の推定}

\begin{lstlisting}[language=C++]
template < typename T >
struct X
{
    X( T t ) { }
} ;

int main()
{
    X x1(0) ; // X<int>
    X x2(0.0) ; // X<double>
    X x3("hello") ; // X<char const *>
}
\end{lstlisting}

これは関数テンプレートが実引数からテンプレート実引数の推定が行えるのと同じだ。

\begin{lstlisting}[language=C++]
template < typename T >
void f( T t ) { }

int main()
{
    f( 0 ) ; // f<int>
    f( 0.0 ) ; // f<double>
    f( "hello" ) ; // f<char const *>
}
\end{lstlisting}

%
% SubSection 3.16.1
\hypersubsection{section3-16-1}{推定ガイド}

クラステンプレートのコンストラクターからの実引数は便利だが、クラスのコンストラクターはクラステンプレートのテンプレートパラメーターに一致しない場合もある。そのような場合はそのままでは実引数推定ができない。
\index{すいていがいど@推定ガイド}

\begin{lstlisting}[language=C++]
// コンテナー風のクラス
template < typename T >
class Container
{
    std::vector<T> c ;
public :
    // 初期化にイテレーターのペアを取る
    // IteratorはTではない
    // Tは推定できない
    template < typename Iterator >
    Container( Iterator first, Iterator last )
        : c( first, last )
    { }
} ;


int main()
{
    int a[] = { 1,2,3,4,5 } ;

    // エラー
    // Tを推定できない
    Container c( std::begin(a), std::end(a) ) ;
}
\end{lstlisting}

このため、C++17には推定ガイドという機能が提供されている。

\begin{lstlisting}[language=C++]
テンプレート名( 引数リスト ) -> テンプレートid ;
\end{lstlisting}

これを使うと、以下のように書ける。

\begin{lstlisting}[language=C++]
template < typename Iterator >
Container( Iterator, Iterator )
-> Container< typename std::iterator_traits< Iterator >::value_type > ;
\end{lstlisting}

C++コンパイラーはこの推定ガイドを使って、\lstinline[breaklines=true]!Container<T>::Container(Iterator, Iterator)!からは、\lstinline!T!を\lstinline!std::iterator_traits<Iterator>::value_type!として推定すればいいのだと判断できる。

たとえば、初期化リストに対応するには以下のように書く。

\begin{lstlisting}[language=C++]
template < typename T >
class Container
{
    std::vector<T> c ;
public :

    Container( std::initializer_list<T> init )
        : c( init )
    { }
} ;


template < typename T >
Container( std::initializer_list<T> ) -> Container<T> ;


int main()
{
    Container c = { 1,2,3,4,5 } ;
}
\end{lstlisting}

C++コンパイラーはこの推定ガイドから、\lstinline[breaklines=true]!Container<T>::Container( std::initializer_list<T> )!の場合は\lstinline!T!を\lstinline!T!として推定すればよいことがわかる。

機能テストマクロは~\lstinline!__cpp_deduction_guides!, 値は201606。
\index{\_\_cpp\_deduction\_guides@\texttt{\_\_cpp\_deduction\_guides}}
