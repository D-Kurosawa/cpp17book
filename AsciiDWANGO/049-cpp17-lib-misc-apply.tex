%
% Section 9.3
\hypersection{section9-3}{apply : tupleの要素を実引数に関数を呼び出す}

\bgroup
\begin{lstlisting}[language=C++]
template <class F, class Tuple>
constexpr decltype(auto) apply(F&& f, Tuple&& t);
\end{lstlisting}
\egroup

\lstinline!std::apply!は\lstinline!tuple!のそれぞれの要素を順番に実引数に渡して関数を呼び出すヘルパー関数だ。
\index{std::apply@\texttt{std::apply}}\index{tuple@\texttt{tuple}}

ある要素数\lstinline!N!の\lstinline!tuple t!と関数オブジェクト\lstinline!f!に対して、\lstinline!apply( f, t )!は、\lstinline!f( get<0>(t), get<1>(t), ... , get<N-1>(t) )!のように\lstinline!f!を関数呼び出しする。

\vskip 1zw
\noindent
\textsf{例}:

\begin{lstlisting}[language=C++]
template < typename ... Types >
void f( Types ... args ) { }

int main()
{
    // int, int, int
    std::tuple t1( 1,2,3 ) ;

    // f( 1, 2, 3 )の関数呼び出し
    std::apply( f, t1 ) ;

    // int, double, const char *
    std::tuple t2( 123, 4.56, "hello" ) ;

    // f( 123, 4.56, "hello" )の関数呼び出し
    std::apply( f, t2 ) ;
}
\end{lstlisting}

