%
% Chapter 6
\hyperchapter{chapter6}{メモリーリソース : 動的ストレージ確保ライブラリ}{メモリーリソース :\\動的ストレージ確保ライブラリ}

ヘッダーファイル~\lstinline!<memory_resource>!~で定義されているメモリーリソースは、動的ストレージを確保するためのC++17で追加されたライブラリだ。その特徴は以下のとおり。
\index{<memory\_resource>@\texttt{<memory\_resource>}}\index{めもりりそす@メモリーリソース}\index{どうてきすとれじ@動的ストレージ}

\begin{itemize}
\itemsep1pt\parskip0pt\parsep0pt
\item
  アロケーターに変わる新しいインターフェースとしてのメモリーリソース
\item
  ポリモーフィックな振る舞いを可能にするアロケーター
\item
  標準で提供されるさまざまな特性を持ったメモリーリソースの実装
\end{itemize}

%
% Section 6.1
\hypersection{section6-1}{メモリーリソース}

メモリーリソースはアロケーターに変わる新しいメモリー確保と解放のためのインターフェースとしての抽象クラスだ。コンパイル時に挙動を変える静的ポリモーフィズム設計のアロケーターと違い、メモリーリソースは実行時に挙動を変える動的ポリモーフィズム設計となっている。
\index{あろけた@アロケーター}

\begin{lstlisting}[language=C++]
void f( memory_resource * mem )
{
    // 10バイトのストレージを確保
    auto ptr = mem->allocate( 10 ) ;
    // 確保したストレージを解放
    mem->deallocate( ptr ) ;
}
\end{lstlisting}

クラス\lstinline!std::pmr::memory_resource!の宣言は以下のとおり。
\index{std::pmr::memory\_resource@\texttt{std::pmr::memory\_resource}}\index{memory\_resource@\texttt{memory\_resource}}

\begin{lstlisting}[language=C++]
namespace std::pmr {

class memory_resource {
public:
    virtual ~ memory_resource();
    void* allocate(size_t bytes, size_t alignment = max_align);
    void deallocate(void* p, size_t bytes,
                    size_t alignment = max_align);
    bool is_equal(const memory_resource& other) const noexcept;

private:
    virtual void* do_allocate(size_t bytes, size_t alignment) = 0;
    virtual void do_deallocate( void* p, size_t bytes,
                                size_t alignment) = 0;
    virtual bool do_is_equal(const memory_resource& other)
        const noexcept = 0;
};

}
\end{lstlisting}

クラス\lstinline!memory_resource!は\lstinline!std::pmr!名前空間スコープの中にある。

%
% SubSection 6.1.1
\hypersubsection{section6-1-1}{メモリーリソースの使い方}

\lstinline!memory_resource!を使うのは簡単だ。\lstinline!memory_resource!のオブジェクトを確保したら、メンバー関数\lstinline!allocate( bytes, alignment )!でストレージを確保する。メンバー関数\lstinline!deallocate( p, bytes, alignment )!でストレージを解放する。
\index{allocate@\texttt{allocate}}\index{memory\_resource@\texttt{memory\_resource}!allocate@\texttt{allocate}}\index{deallocate@\texttt{deallocate}}\index{memory\_resource@\texttt{memory\_resource}!deallocate@\texttt{deallocate}}

\begin{lstlisting}[language=C++]
void f( std::pmr::memory_resource * mem )
{
    // 100バイトのストレージを確保
    void * ptr = mem->allocate( 100 ) ;
    // ストレージを解放
    mem->deallocate( ptr, 100 ) ;
}
\end{lstlisting}

2つの\lstinline!memory_resource!のオブジェクト\lstinline!a!,
\lstinline!b!があるとき、一方のオブジェクトで確保したストレージをもう一方のオブジェクトで解放できるとき、\lstinline!a.is_equal( b )!は\lstinline!true!を返す。
\index{is\_equal@\texttt{is\_equal}}\index{memory\_resource@\texttt{memory\_resource}!is\_equal@\texttt{is\_equal}}

\begin{lstlisting}[language=C++]
void f( std::pmr::memory_resource * a, std::pmr::memory_resource * b )
{
    void * ptr = a->allocate( 1 ) ;

    // aで確保したストレージはbで解放できるか?
    if ( a->is_equal( *b ) )
    {// できる
        b->deallocate( ptr, 1 ) ;
    }
    else
    {// できない
        a->deallocate( ptr, 1 ) ;
    }
}
\end{lstlisting}

\lstinline!is_equal!を呼び出す~\lstinline!operator ==!~と~\lstinline"operator !="~も提供されている。

\begin{lstlisting}[language=C++]
void f( std::pmr::memory_resource * a, std::pmr::memory_resource * b )
{
    bool b1 = ( *a == *b ) ;
    bool b2 = ( *a != *b ) ;
}
\end{lstlisting}

%
% SubSection 6.1.2
\hypersubsection{section6-1-2}{メモリーリソースの作り方}

独自のメモリーアロケーターを\lstinline!memory_resource!のインターフェースに合わせて作るには、\lstinline!memory_resource!から派生した上で、\lstinline!do_allocate!,
\lstinline!do_deallocate!,
\lstinline!do_is_equal!の3つの\lstinline!private!純粋\lstinline!virtual!メンバー関数をオーバーライドする。必要に応じてデストラクターもオーバーライドする。
\index{do\_allocate@\texttt{do\_allocate}}\index{memory\_resource@\texttt{memory\_resource}!do\_allocate@\texttt{do\_allocate}}\index{do\_deallocate@\texttt{do\_deallocate}}\index{memory\_resource@\texttt{memory\_resource}!do\_deallocate@\texttt{do\_deallocate}}\index{do\_is\_equal@\texttt{do\_is\_equal}}\index{memory\_resource@\texttt{memory\_resource}!do\_is\_equal@\texttt{do\_is\_equal}}

\begin{lstlisting}[language=C++]
class memory_resource {
    // 非公開
    static constexpr size_t max_align = alignof(max_align_t);

public:
    virtual ~ memory_resource();

private:
    virtual void* do_allocate(size_t bytes, size_t alignment) = 0;
    virtual void do_deallocate( void* p, size_t bytes,
                                size_t alignment) = 0;
    virtual bool do_is_equal(const memory_resource& other)
        const noexcept = 0;
};
\end{lstlisting}

\lstinline!do_allocate(bytes, alignment)!は少なくとも\lstinline!alignment!バイトでアライメントされた\lstinline!bytes!バイトのストレージへのポインターを返す。ストレージが確保できなかった場合は、適切な例外を\lstinline!throw!する。

\lstinline!do_deallocate(p, bytes, alignment)!は事前に同じ~\lstinline!*this!~から呼び出された\lstinline!allocate( bytes, alignment )!で返されたポインター\lstinline!p!を解放する。すでに解放されたポインター\lstinline!p!を渡してはならない。例外は投げない。

\lstinline!do_is_equal(other)!は、\lstinline!*thisとother!が互いに一方で確保したストレージをもう一方で解放できる場合に\lstinline!true!を返す。

たとえば、\lstinline!malloc!/\lstinline!free!を使った\lstinline!memory_resource!の実装は以下のとおり。
\index{malloc@\texttt{malloc}}\index{memory\_resource@\texttt{memory\_resource}!malloc@\texttt{malloc}}\index{free@\texttt{free}}\index{memory\_resource@\texttt{memory\_resource}!free@\texttt{free}}

\begin{lstlisting}[language=C++]
// malloc/freeを使ったメモリーリソース
class malloc_resource : public std::pmr::memory_resource
{
public :
    //
    ~malloc_resource() { }
private :
    // ストレージの確保
    // 失敗した場合std::bad_allocをthrowする
    virtual void * 
    do_allocate( std::size_t bytes, std::size_t alignment ) override
    {
        void * ptr = std::malloc( bytes ) ;
        if ( ptr == nullptr )
        { throw std::bad_alloc{} ; }

        return ptr ;
    }

    // ストレージの解放
    virtual void 
    do_deallocate(  void * p, std::size_t bytes, 
                    std::size_t alignment ) override
    {
        std::free( p ) ;
    }

    virtual bool 
    do_is_equal( const memory_resource & other )
        const noexcept override
    {
        return dynamic_cast< const malloc_resource * >
                    ( &other ) != nullptr ;
    }

} ;
\end{lstlisting}

\lstinline!do_allocate!は\lstinline!malloc!でストレージを確保し、\lstinline!do_deallocate!は\lstinline!free!でストレージを解放する。メモリーリソースで0バイトのストレージを確保しようとしたときの規定はないので、\lstinline!malloc!の挙動に任せる。\lstinline!malloc!は0バイトのメモリーを確保しようとしたとき、C11では規定がない。POSIXでは\lstinline!null!ポインターを返すか、\lstinline!free!で解放可能な何らかのアドレスを返すものとしている。

\lstinline!do_is_equal!は、\lstinline!malloc_resource!でさえあればどのオブジェクトから確保されたストレージであっても解放できるので、\lstinline!*this!が\lstinline!malloc_resource!であるかどうかを\lstinline!dynamic_cast!で確認している。

%
% Section 6.2
\hypersection{section6-2}{polymorphic\texttt{\_}allocator : 動的ポリモーフィズムを実現するアロケーター}
\index{std::pmr::polymorphic\_allocator@\texttt{std::pmr::polymorphic\_allocator}}\index{polymorphic\_allocator@\texttt{polymorphic\_allocator}}\index{どうてきぽりもふいずむ@動的ポリモーフィズム}

\lstinline!std::pmr::polymorphic_allocator!はメモリーリソースを動的ポリモーフィズムとして振る舞うアロケーターにするためのライブラリだ。

従来のアロケーターは、静的ポリモーフィズムを実現するために設計されていた。たとえば独自の\lstinline!custom_int_allocator!型を使いたい場合は以下のように書く。

\begin{lstlisting}[language=C++]
std::vector< int, custom_int_allocator > v ;
\end{lstlisting}

コンパイル時に使うべきアロケーターが決定できる場合はこれでいいのだが、実行時にアロケーターを選択したい場合、アロケーターをテンプレート引数に取る設計は問題になる。

そのため、C++17ではメモリーリソースをコンストラクター引数に取り、メモリーリソースからストレージを確保する実行時ポリモーフィックの振る舞いをする\lstinline!std::pmr::polymorphic_allocator!が追加された。

たとえば、標準入力から\lstinline!true!か\lstinline!false!が入力されたかによって、システムのデフォルトのメモリーリソースと、\lstinline!monotonic_buffer_resource!を実行時に切り替えるには、以下のように書ける。
\index{monotonic\_buffer\_resource@\texttt{monotonic\_buffer\_resource}}

\begin{lstlisting}[language=C++]
int main()
{
    bool b;

    std::cin >> b ;

    std::pmr::memory_resource * mem ;
    std::unique_ptr< memory_resource > mono ;

    if ( b )
    { // デフォルトのメモリーリソースを使う
        mem = std::pmr::get_default_resource() ;
    }
    else
    { // モノトニックバッファーを使う
        mono = std::make_unique< std::pmr::monotonic_buffer_resource >
                ( std::pmr::get_default_resource() ) ;
        mem = mono.get() ;
    }

    std::vector< int, std::pmr::polymorphic_allocator<int> >
        v( std::pmr::polymorphic_allocator<int>( mem ) ) ;
}
\end{lstlisting}

\lstinline!std::pmr::polymorphic_allocator!は以下のように宣言されている。

\begin{lstlisting}[language=C++]
namespace std::pmr {

template <class T>
class polymorphic_allocator ;

}
\end{lstlisting}

テンプレート実引数には~\lstinline!std::allocator<T>!~と同じく、確保する型を与える。

%
% SubSection 6.2.1
\hypersubsection{section6-2-1}{コンストラクター}
\index{polymorphic\_allocator@\texttt{polymorphic\_allocator}!こんすとらくた@コンストラクター}

\bgroup
\begin{lstlisting}[language=C++]
polymorphic_allocator() noexcept;
polymorphic_allocator(memory_resource* r);
\end{lstlisting}
\egroup

\lstinline!std::pmr::polymorphic_allocator!のデフォルトコンストラクターは、メモリーリソースを\lstinline!std::pmr::get_default_resource()!で取得する。
\index{get\_default\_resource@\texttt{get\_default\_resource}}

\lstinline!memory_resource *!~を引数に取るコンストラクターは、渡されたメモリーリソースをストレージ確保に使う。\lstinline!polymorphic_allocator!の生存期間中、メモリーリソースへのポインターは妥当なものでなければならない。

\begin{lstlisting}[language=C++]
int main()
{
    // p1( std::pmr::get_default_resource () ) と同じ
    std::pmr::polymorphic_allocator<int> p1 ;

    std::pmr::polymorphic_allocator<int> p2(
        std::pmr::get_default_resource() ) ;
}
\end{lstlisting}

後は通常のアロケーターと同じように振る舞う。

%
% Section 6.3
\hypersection{section6-3}{プログラム全体で使われるメモリーリソースの取得}
\index{めもりりそす@メモリーリソース!しゆとく@取得}

C++17では、プログラム全体で使われるメモリーリソースへのポインターを取得することができる。

%
% SubSection 6.3.1
\hypersubsection{section6-3-1}{new\texttt{\_}delete\texttt{\_}resource()}
\index{new\_delete\_resource@\texttt{new\_delete\_resource}}\index{memory\_resource@\texttt{memory\_resource}!new\_delete\_resource@\texttt{new\_delete\_resource}}

\bgroup
\begin{lstlisting}[language=C++]
memory_resource* new_delete_resource() noexcept ;
\end{lstlisting}
\egroup

関数\lstinline!new_delete_resource!はメモリーリソースへのポインターを返す。参照されるメモリーリソースは、ストレージの確保に~\lstinline!::operator new!~を使い、ストレージの解放に~\lstinline!::operator delete!~を使う。

\begin{lstlisting}[language=C++]
int main()
{
    auto mem = std::pmr::new_delete_resource() ;
}
\end{lstlisting}

%
% SubSection 6.3.2
\hypersubsection{section6-3-2}{null\texttt{\_}memory\texttt{\_}resource()}
\index{null\_memory\_resource@\texttt{null\_memory\_resource}}\index{memory\_resource@\texttt{memory\_resource}!null\_memory\_resource@\texttt{null\_memory\_resource}}

\bgroup
\begin{lstlisting}[language=C++]
memory_resource* null_memory_resource() noexcept ;
\end{lstlisting}
\egroup

関数\lstinline!null_memory_resource!はメモリーリソースへのポインターを返す。参照されるメモリーリソースの\lstinline!allocate!は必ず失敗し、\lstinline!std::bad_alloc!を\lstinline!throw!する。\lstinline!deallocate!は何もしない。

このメモリーリソースは、ストレージの確保に失敗した場合のコードをテストする目的で使える。

%
% SubSection 6.3.3
\hypersubsection{section6-3-3}{デフォルトリソース}

\bgroup
\begin{lstlisting}[language=C++]
memory_resource* set_default_resource(memory_resource* r) noexcept ;
memory_resource* get_default_resource() noexcept ;
\end{lstlisting}
\egroup

デフォルト・メモリーリソース・ポインターとは、メモリーリソースを明示的に指定することができない場合に、システムがデフォルトで利用するメモリーリソースへのポインターのことだ。初期値は\lstinline!new_delete_resource()!の戻り値となっている。

現在のデフォルト・メモリーリソース・ポインターと取得するためには、関数\lstinline!get_default_resource!を使う。デフォルト・メモリーリソース・ポインターを独自のメモリーリソースに差し替えるには、関数\lstinline!set_default_resource!を使う。
\index{get\_default\_resource@\texttt{get\_default\_resource}}\index{memory\_resource@\texttt{memory\_resource}!get\_default\_resource@\texttt{get\_default\_resource}}\index{set\_default\_resource@\texttt{set\_default\_resource}}\index{memory\_resource@\texttt{memory\_resource}!set\_default\_resource@\texttt{set\_default\_resource}}

\begin{lstlisting}[language=C++]
int main()
{
    // 現在のデフォルトのメモリーリソースへのポインター
    auto init_mem = std::pmr::get_default_resource() ;

    std::pmr::synchronized_pool_resource pool_mem ;

    // デフォルトのメモリーリソースを変更する
    std::pmr::set_default_resource( &pool_mem ) ;

    auto current_mem = std::pmr::get_default_resource() ;

    // true
    bool b = current_mem == pool_mem ;
}
\end{lstlisting}

%
% Section 6.4
\hypersection{section6-4}{標準ライブラリのメモリーリソース}

標準ライブラリはメモリーリソースの実装として、プールリソースとモノトニックリソースを提供している。このメモリーリソースの詳細は後に解説するが、ここではそのための事前知識として、汎用的なメモリーアロケーター一般の解説をする。
\index{めもりりそす@メモリーリソース}\index{めもりあろけた@メモリーアロケーター}

プログラマーはメモリーを気軽に確保している。たとえば47バイトとか151バイトのような中途半端なサイズのメモリーを以下のように気軽に確保している。

\begin{lstlisting}[language=C++]
int main()
{
    auto mem = std::get_default_resource() ;

    auto p1 = mem->allocate( 47 ) ;
    auto p2 = mem->allocate( 151 ) ;

    mem->deallocate( p1 ) ;
    mem->deallocate( p2 ) ;
}
\end{lstlisting}

しかし、残念ながら現実のハードウェアやOSのメモリー管理は、このように柔軟にはできていない。たとえば、あるアーキテクチャーとOSでは、メモリーはページサイズと呼ばれる単位でしか確保できない。そして最小のページサイズですら4Kバイトであったりする。もしシステムの低級なメモリー管理を使って上のコードを実装しようとすると、47バイト程度のメモリーを使うのに3Kバイト超の無駄が生じることになる。

他にもアライメントの問題がある。アーキテクチャーによってはメモリーアドレスが適切なアライメントに配置されていないとメモリーアクセスができないか、著しくパフォーマンスが落ちることがある。
\index{あらいめんと@アライメント}

\lstinline!malloc!や\lstinline!operator new!などのメモリーアロケーターは、低級なメモリー管理を隠匿し、小さなサイズのメモリー確保を効率的に行うための実装をしている。
\index{malloc@\texttt{malloc}}\index{new@\texttt{new}}

一般的には、大きな連続したアドレス空間のメモリーを確保し、その中に管理用のデータ構造を作り、メモリーを必要なサイズに切り出す。

\begin{lstlisting}[language=C++]
// 実装イメージ

// ストレージを分割して管理するためのリンクリストデータ構造
struct alignas(std::max_align_t) chunk
{
    chunk * next ;
    chunk * prev ;
    std::size_t size ;
} ;

class memory_allocator : public std::pmr::memory_resource
{
    chunk * ptr ; // ストレージの先頭へのポインター
    std::size_t size ; // ストレージのサイズ
    std::mutex m ; // 同期用

    
public :

    memory_allocator()
    {
        // 大きな連続したストレージを確保
    }

    virtual void * 
    do_allocate( std::size_t bytes, std::size_t alignment ) override
    {
        std::scoped_lock lock( m ) ; 
        // リンクリストをたどり、十分な大きさの未使用領域を探し、リンクリスト構造体を
        // 構築して返す
        // アライメント要求に注意
    }

    virtual void * 
    do_allocate( std::size_t bytes, std::size_t alignment ) override
    {
        std::scoped_lock lock( m ) ;
        // リンクリストから該当する部分を削除
    }

    virtual bool 
    do_is_equal( const memory_resource & other )
        const noexcept override
    { 
    // *thisとotherで相互にストレージを解放できるかどうか返す
    }
} ;
\end{lstlisting}

%
% Section 6.5
\hypersection{section6-5}{プールリソース}

プールリソースはC++17の標準ライブラリが提供しているメモリーリソースの実装だ。\lstinline!synchronized_pool_resource!と\lstinline!unsynchronized_pool_resource!の2つがある。
\index{ぷるりそす@プールリソース}\index{めもりりそす@メモリーリソース!ぷるりそす@プールリソース}\index{ぷるりそす@プールリソース}\index{synchronized\_pool\_resource@\texttt{synchronized\_pool\_resource}}\index{ぷるりそす@プールリソース!synchronized\_pool\_resource@\texttt{synchronized\_pool\_resource}}\index{unsynchronized\_pool\_resource@\texttt{unsynchronized\_pool\_resource}}\index{ぷるりそす@プールリソース!unsynchronized\_pool\_resource@\texttt{unsynchronized\_pool\_resource}}

%
% SubSection 6.5.1
\hypersubsection{section6-5-1}{アルゴリズム}

プールリソースは以下のような特徴を持つ。

\begin{itemize}[leftmargin=*]
\item
  プールリソースのオブジェクトが破棄されるとき、そのオブジェクトから\lstinline!allocate!で確保したストレージは、明示的に\lstinline!deallocate!を呼ばずとも解放される。
\end{itemize}

\begin{lstlisting}[language=C++]
void f()
{
    std::pmr::synchronized_pool_resource mem ;
    mem.allocate( 10 ) ;

    // 確保したストレージは破棄される
}
\end{lstlisting}

\begin{itemize}[leftmargin=*]
\item
  プールリソースの構築時に、上流メモリーリソースを与えることができる。プールリソースは上流メモリーリソースからチャンクのためのストレージを確保する。
\end{itemize}

\begin{lstlisting}[language=C++]
int main()
{
    // get_default_resource()が使われる
    std::pmr::synchronized_pool_resource m1 ;

    // 独自の上流メモリーリソースを指定
    custom_memory_resource mem ;
    std::pmr::synchronized_pool_resource m2( &mem ) ;
    
}
\end{lstlisting}

\begin{itemize}[leftmargin=*]
\item
  プールリソースはストレージを確保する上流メモリーリソースから、プールと呼ばれる複数のストレージを確保する。プールは複数のチャンクを保持している。チャンクは複数の同一サイズのブロックを保持している。プールリソースに対する\lstinline!do_allocate(size, alignment)!は、少なくとも\lstinline!size!バイトのブロックサイズのプールのいずれかのチャンクのブロックが割り当てられる。
  \index{do\_allocate@\texttt{do\_allocate}}\index{memory\_resource@\texttt{memory\_resource}!do\_allocate@\texttt{do\_allocate}}

  もし、最大のブロックサイズを超えるサイズのストレージを確保しようとした場合、上流メモリーリソースから確保される。
\end{itemize}

\begin{lstlisting}[language=C++]
// 実装イメージ

namespace std::pmr {

// チャンクの実装
template < size_t block_size >
class chunk
{
    blocks<block_size> b ;
}

// プールの実装
template < size_t block_size >
class pool : public memory_resource
{
    chunks<block_size> c ;
} ;

class pool_resource : public memory_resource
{
    // それぞれのブロックサイズのプール
    pool<8> pool_8bytes ;
    pool<16> pool_16bytes ;
    pool<32> pool_32bytes ;

    // 上流メモリーリソース
    memory_resource * mem ;


    virtual void * do_allocate( size_t bytes, size_t alignment ) override
    {
        // 対応するブロックサイズのプールにディスパッチ
        if ( bytes <= 8 )
            return pool_8bytes.allocate( bytes, alignment ) ;
        else if ( bytes <= 16 )
            return pool_16bytes.allocate( bytes, alignment ) ;
        else if ( bytes < 32 )
            return pool_32bytes.allocate( bytes, alignment ) ;
        else
        // 最大ブロックサイズを超えたので上流メモリーリソースにディスパッチ
            return mem->allocate( bytes, alignment ) ;
    }
} ;

}
\end{lstlisting}

\begin{itemize}[leftmargin=*]
\item
  プールリソースは構築時に\lstinline!pool_options!を渡すことにより、最大ブロックサイズと最大チャンクサイズを設定できる。
  \index{pool\_options@\texttt{pool\_options}}\index{ぷるりそす@プールリソース!pool\_options@\texttt{pool\_options}}
\item
  マルチスレッドから呼び出しても安全な同期を取る\lstinline!synchronized_pool_resource!と、同期を取らない\lstinline!unsynchronized_pool_resource!がある。
\end{itemize}

%
% SubSection 6.5.2
\hypersubsection{section6-5-2}{synchronized/unsynchronized\texttt{\_}pool\texttt{\_}resource}

プールリソースには、\lstinline!synchronized_pool_resource!と\lstinline[breaklines=true]!unsynchronized_pool _resource!がある。どちらもクラス名以外は同じように使える。ただし、\lstinline!synchronized_pool_resource!は複数のスレッドから同時に実行しても使えるように内部で同期が取られているのに対し、\lstinline!unsynchronized_pool_resource!は同期を行わない。\lstinline!unsynchronized_pool_resource!は複数のスレッドから同時に呼び出すことはできない。
\index{synchronized\_pool\_resource@\texttt{synchronized\_pool\_resource}}\index{ぷるりそす@プールリソース!synchronized\_pool\_resource@\texttt{synchronized\_pool\_resource}}\index{unsynchronized\_pool\_resource@\texttt{unsynchronized\_pool\_resource}}\index{ぷるりそす@プールリソース!unsynchronized\_pool\_resource@\texttt{unsynchronized\_pool\_resource}}

\begin{lstlisting}[language=C++]
// 実装イメージ

namespace std::pmr {

class synchronized_pool_resource : public memory_resource
{
    std::mutex m ;

    virtual void * 
    do_allocate( size_t size, size_t alignment ) override
    {
        // 同期する
        std::scoped_lock l(m) ;
        return do_allocate_impl( size, alignment ) ;
    }
} ;

class unsynchronized_pool_resource : public memory_resource
{
    virtual void * 
    do_allocate( size_t size, size_t alignment ) override
    {
        // 同期しない
        return do_allocate_impl( size, alignment ) ;
    }
} ;

}
\end{lstlisting}

%
% SubSection 6.5.3
\hypersubsection{section6-5-3}{pool\texttt{\_}options}

\lstinline!pool_options!はプールリソースの挙動を指定するためのクラスで、以下のように定義されている。
\index{pool\_options@\texttt{pool\_options}}\index{ぷるりそす@プールリソース!pool\_options@\texttt{pool\_options}}

\begin{lstlisting}[language=C++]
namespace std::pmr {

struct pool_options {
    size_t max_blocks_per_chunk = 0;
    size_t largest_required_pool_block = 0;
};

}
\end{lstlisting}

このクラスのオブジェクトをプールリソースのコンストラクターに与えることで、プールリソースの挙動を指定できる。ただし、\lstinline!pool_options!による指定はあくまでも目安で、実装には従う義務はない。

\lstinline!max_blocks_per_chunk!は、上流メモリーリソースからプールのチャンクを補充する際に一度に確保する最大のブロック数だ。この値がゼロか、実装の上限より大きい場合、実装の上限が使われる。実装は指定よりも小さい値を使うことができるし、またプールごとに別の値を使うこともできる。
\index{max\_blocks\_per\_chunk@\texttt{max\_blocks\_per\_chunk}}\index{ぷるりそす@プールリソース!max\_blocks\_per\_chunk@\texttt{max\_blocks\_per\_chunk}}

\lstinline!largest_required_pool_block!はプール機構によって確保される最大のストレージのサイズだ。この値より大きなサイズのストレージを確保しようとすると、上流メモリーストレージから直接確保される。この値がゼロか、実装の上限よりも大きい場合、実装の上限が使われる。実装は指定よりも大きい値を使うこともできる。
\index{largest\_required\_pool\_block@\texttt{largest\_required\_pool\_block}}\index{ぷるりそす@プールリソース!largest\_required\_pool\_block@\texttt{largest\_required\_pool\_block}}

%
% SubSection 6.5.4
\hypersubsection{section6-5-4}{プールリソースのコンストラクター}
\index{ぷるりそす@プールリソース!こんすとらくた@コンストラクター}

プールリソースの根本的なコンストラクターは以下のとおり。\lstinline!synchronized!と\lstinline!unsynchronized!どちらも同じだ。

\begin{lstlisting}[language=C++]
pool_resource(const pool_options& opts, memory_resource* upstream);

pool_resource()
: pool_resource(pool_options(), get_default_resource()) {}
explicit pool_resource(memory_resource* upstream)
: pool_resource(pool_options(), upstream) {}
explicit pool_resource(const pool_options& opts)
: pool_resource(opts, get_default_resource()) {}
\end{lstlisting}

\lstinline!pool_options!と~\lstinline!memory_resource *!~を指定する。指定しない場合はデフォルト値が使われる。

%
% SubSection 6.5.5
\hypersubsection{section6-5-5}{プールリソースのメンバー関数}

%
% SubsubSection 6.5.5.1
\hypersubsubsection{section6-5-5-1}{release()}
\index{release@\texttt{release}}\index{ぷるりそす@プールリソース!release@\texttt{release}}

\bgroup
\begin{lstlisting}[language=C++]
void release();
\end{lstlisting}
\egroup

確保したストレージすべてを解放する。たとえ明示的に\lstinline!deallocate!を呼び出されていないストレージも解放する。

\begin{lstlisting}[language=C++]
int main()
{
    synchronized_pool_resource mem ;
    void * ptr = mem.allocate( 10 ) ;

    // ptrは解放される
    mem.release() ;

}
\end{lstlisting}

%
% SubsubSection 6.5.5.2
\vskip 1zw
\hypersubsubsection{section6-5-5-2}{upstream\texttt{\_}resource()}
\index{upstream\_resource@\texttt{upstream\_resource}}\index{ぷるりそす@プールリソース!upstream\_resource@\texttt{upstream\_resource}}

\bgroup
\begin{lstlisting}[language=C++]
memory_resource* upstream_resource() const;
\end{lstlisting}
\egroup

構築時に渡した上流メモリーリソースへのポインターを返す。

%
% SubsubSection 6.5.5.3
\vskip 1zw
\hypersubsubsection{section6-5-5-3}{options()}
\index{options@\texttt{options}}\index{ぷるりそす@プールリソース!options@\texttt{options}}

\bgroup
\begin{lstlisting}[language=C++]
pool_options options() const;
\end{lstlisting}
\egroup

構築時に渡した\lstinline!pool_options!オブジェクトと同じ値を返す。

%
% Section 6.6
\hypersection{section6-6}{モノトニックバッファーリソース}

モノトニックバッファーリソースはC++17で標準ライブラリに追加されたメモリーリソースの実装だ。クラス名は\lstinline!monotonic_buffer_resource!。
\index{ものとにつくばつふありそす@モノトニックバッファーリソース}\index{めもりりそす@メモリーリソース!ものとにつくばつふありそす@モノトニックバッファーリソース}\index{monotonic\_buffer\_resource@\texttt{monotonic\_buffer\_resource}}\index{ものとにつくばつふありそす@モノトニックバッファーリソース!monotonic\_buffer\_resource@\texttt{monotonic\_buffer\_resource}}

モノトニックバッファーリソースは高速にメモリーを確保し、一気に解放するという用途に特化した特殊な設計をしている。モノトニックバッファーリソースはメモリー解放をせず、メモリー使用量がモノトニックに増え続けるので、この名前が付いている。

たとえばゲームで1フレームを描画する際に大量に小さなオブジェクトのためのストレージを確保し、その後確保したストレージをすべて解放したい場合を考える。通常のメモリーアロケーターでは、メモリー片を解放するためにメモリー全体に構築されたデータ構造をたどり、データ構造を書き換えなければならない。この処理は高くつく。すべてのメモリー片を一斉に解放してよいのであれば、データ構造をいちいちたどったり書き換えたりする必要はない。メモリーの管理は、単にポインターだけでよい。

\begin{lstlisting}[language=C++]
// 実装イメージ

namespace std::pmr {

class monotonic_buffer_resource : public memory_resource
{
    // 連続した長大なストレージの先頭へのポインター
    void * ptr ;
    // 現在の未使用ストレージの先頭へのポインター
    std::byte * current ;

    virtual void * 
    do_allocate( size_t bytes, size_t alignment ) override
    {
        void * result = static_cast<void *>(current) ;
        current += bytes ; // 必要であればアライメント調整
        return result ;
    }

    virtual void 
    do_deallocate( void * ptr, size_t bytes, size_t alignment ) override 
    {
        // 何もしない
    }

public :
    ~monotonic_buffer_resource()
    {
        // ptrの解放
    }
} ;

}
\end{lstlisting}

このように、基本的な実装としては、\lstinline!do_allocate!はポインターを加算して管理するだけだ。なぜならば解放処理がいらないため、個々のストレージ片を管理するためのデータ構造を構築する必要がない。\lstinline!do_deallocate!は何もしない。デストラクターはストレージ全体を解放する。
\index{do\_allocate@\texttt{do\_allocate}}\index{ものとにつくばつふありそす@モノトニックバッファーリソース!do\_allocate@\texttt{do\_allocate}}\index{do\_deallocate@\texttt{do\_deallocate}}\index{ものとにつくばつふありそす@モノトニックバッファーリソース!do\_deallocate@\texttt{do\_deallocate}}

%
% SubSection 6.6.1
\hypersubsection{section6-6-1}{アルゴリズム}

モノトニックバッファーリソースは以下のような特徴を持つ。

\begin{itemize}[leftmargin=*]
\item
  \lstinline!deallocate!呼び出しは何もしない。メモリー使用量はリソースが破棄されるまでモノトニックに増え続ける。
\end{itemize}
\index{deallocate@\texttt{deallocate}}\index{ものとにつくばつふありそす@モノトニックバッファーリソース!deallocate@\texttt{deallocate}}

\begin{lstlisting}[language=C++]
int main()
{
    std::pmr::monotonic_buffer_resource mem ;

    void * ptr = mem.allocate( 10 ) ;
    // 何もしない
    // ストレージは解放されない
    mem.deallocate( ptr ) ;

    // memが破棄される際に確保したストレージはすべて破棄される
}
\end{lstlisting}

\begin{itemize}[leftmargin=*]
\item
  メモリー確保に使う初期バッファーを与えることができる。ストレージ確保の際に、初期バッファーに空きがある場合はそこから確保する。空きがない場合は上流メモリーリソースからバッファーを確保して、バッファーから確保する。
\end{itemize}
\index{allocate@\texttt{allocate}}\index{ものとにつくばつふありそす@モノトニックバッファーリソース!allocate@\texttt{allocate}}

\begin{lstlisting}[language=C++]
int main()
{
    std::byte initial_buffer[10] ;
    std::pmr::monotonic_buffer_resource 
        mem( initial_buffer, 10, std::pmr::get_default_resource() ) ;

    // 初期バッファーから確保
    mem.allocate( 1 ) ;
    // 上流メモリーリソースからストレージを確保して切り出して確保
    mem.allocate( 100 ) ;
    // 前回のストレージ確保で空きがあればそこから
    // なければ新たに上流から確保して切り出す
    mem.allocate( 100 ) ;
}
\end{lstlisting}

\begin{itemize}[leftmargin=*]
\item
  1つのスレッドから使うことを前提に設計されている。\lstinline!allocate!と\lstinline!deallocate!は同期しない。
\item
  メモリーリソースが破棄されると確保されたすべてのストレージも解放される。明示的に\lstinline!deallocate!を呼ばなくてもよい。
\end{itemize}

%
% SubSection 6.6.2
\hypersubsection{section6-6-2}{コンストラクター}
\index{ものとにつくばつふありそす@モノトニックバッファーリソース!こんすとらくた@コンストラクター}

モノトニックバッファーリソースには以下のコンストラクターがある。

\begin{lstlisting}[language=C++]
explicit monotonic_buffer_resource(memory_resource *upstream);
monotonic_buffer_resource(  size_t initial_size,
                            memory_resource *upstream);
monotonic_buffer_resource(  void *buffer, size_t buffer_size,
                            memory_resource *upstream);


monotonic_buffer_resource()
    : monotonic_buffer_resource(get_default_resource()) {}
explicit monotonic_buffer_resource(size_t initial_size)
    : monotonic_buffer_resource(initial_size,
                                get_default_resource()) {}
monotonic_buffer_resource(void *buffer, size_t buffer_size)
    : monotonic_buffer_resource(buffer, buffer_size,
                                get_default_resource()) {}
\end{lstlisting}

初期バッファーを取らないコンストラクターは以下のとおり。

\begin{lstlisting}[language=C++]
explicit monotonic_buffer_resource(memory_resource *upstream);
monotonic_buffer_resource(  size_t initial_size,
                            memory_resource *upstream);

monotonic_buffer_resource()
    : monotonic_buffer_resource(get_default_resource()) {}
explicit monotonic_buffer_resource(size_t initial_size)
    : monotonic_buffer_resource(initial_size,
                                get_default_resource()) {}
\end{lstlisting}

\lstinline!initial_size!は、上流メモリーリソースから最初に確保するバッファーのサイズ(初期サイズ)のヒントとなる。実装はこのサイズか、あるいは実装依存のサイズをバッファーとして確保する。

デフォルトコンストラクターは上流メモリーリソースに\lstinline[breaklines=true]!std::pmr_get_default_ resource()!を与えたのと同じ挙動になる。

\lstinline!size_t!1つだけを取るコンストラクターは、初期サイズだけを与えて後はデフォルトの扱いになる。

初期バッファーを取るコンストラクターは以下のとおり。

\begin{lstlisting}[language=C++]
monotonic_buffer_resource(  void *buffer, size_t buffer_size,
                            memory_resource *upstream);

monotonic_buffer_resource(void *buffer, size_t buffer_size)
    : monotonic_buffer_resource(buffer, buffer_size,
                                get_default_resource()) {}
\end{lstlisting}

初期バッファーは先頭アドレスを~\lstinline!void *!~型で渡し、そのサイズを\lstinline!size_t!型で渡す。

%
% SubSection 6.6.3
\hypersubsection{section6-6-3}{その他の操作}

%
% SubsubSection 6.6.3.1
\hypersubsubsection{section6-6-3-1}{release()}
\index{release@\texttt{release}}\index{ものとにつくばつふありそす@モノトニックバッファーリソース!release@\texttt{release}}

\bgroup
\begin{lstlisting}[language=C++]
void release() ;
\end{lstlisting}
\egroup

メンバー関数\lstinline!release!は、上流リソースから確保されたストレージをすべて解放する。明示的に\lstinline!deallocate!を呼び出していないストレージも解放される。

\begin{lstlisting}[language=C++]
int main()
{
    std::pmr::monotonic_buffer_resource mem ;

    mem.allocate( 10 ) ;

    // ストレージはすべて解放される
    mem.release() ;

}
\end{lstlisting}

%
% SubsubSection 6.6.3.2
\vskip 1zw
\hypersubsubsection{section6-6-3-2}{upstream\texttt{\_}resource()}
\index{upstream\_resource@\texttt{upstream\_resource}}\index{ものとにつくばつふありそす@モノトニックバッファーリソース!upstream\_resource@\texttt{upstream\_resource}}

\bgroup
\begin{lstlisting}[language=C++]
memory_resource* upstream_resource() const;
\end{lstlisting}
\egroup

メンバー関数\lstinline!upstream_resource!は、構築時に与えられた上流メモリーリソースへのポインターを返す。
